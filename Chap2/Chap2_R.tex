% Options for packages loaded elsewhere
\PassOptionsToPackage{unicode}{hyperref}
\PassOptionsToPackage{hyphens}{url}
\PassOptionsToPackage{dvipsnames,svgnames,x11names}{xcolor}
%
\documentclass[
  8pt,
  ignorenonframetext,
]{beamer}
\title{Programmation sous R\\
Chapitre 2: Graphiques - Statistiques}
\author{Mohamed Essaied Hamrita\\
\href{mailto:mhamrita@gmail.com}{\nolinkurl{mhamrita@gmail.com}}\\
\href{https:github.com/Hamrita}{github.com/Hamrita}}
\date{2023-2024}
\institute{Université de Sousse - Tunisie}

\usepackage{pgfpages}
\setbeamertemplate{caption}[numbered]
\setbeamertemplate{caption label separator}{: }
\setbeamercolor{caption name}{fg=normal text.fg}
\beamertemplatenavigationsymbolsempty
% Prevent slide breaks in the middle of a paragraph
\widowpenalties 1 10000
\raggedbottom
\setbeamertemplate{part page}{
  \centering
  \begin{beamercolorbox}[sep=16pt,center]{part title}
    \usebeamerfont{part title}\insertpart\par
  \end{beamercolorbox}
}
\setbeamertemplate{section page}{
  \centering
  \begin{beamercolorbox}[sep=12pt,center]{part title}
    \usebeamerfont{section title}\insertsection\par
  \end{beamercolorbox}
}
\setbeamertemplate{subsection page}{
  \centering
  \begin{beamercolorbox}[sep=8pt,center]{part title}
    \usebeamerfont{subsection title}\insertsubsection\par
  \end{beamercolorbox}
}
\AtBeginPart{
  \frame{\partpage}
}
\AtBeginSection{
  \ifbibliography
  \else
    \frame{\sectionpage}
  \fi
}
\AtBeginSubsection{
  \frame{\subsectionpage}
}
\usepackage{amsmath,amssymb}
\usepackage{lmodern}
\usepackage{iftex}
\ifPDFTeX
  \usepackage[T1]{fontenc}
  \usepackage[utf8]{inputenc}
  \usepackage{textcomp} % provide euro and other symbols
\else % if luatex or xetex
  \usepackage{unicode-math}
  \defaultfontfeatures{Scale=MatchLowercase}
  \defaultfontfeatures[\rmfamily]{Ligatures=TeX,Scale=1}
\fi
\usetheme[]{Frankfurt}
\usecolortheme{dolphin}
\usefonttheme{professionalfonts}
% Use upquote if available, for straight quotes in verbatim environments
\IfFileExists{upquote.sty}{\usepackage{upquote}}{}
\IfFileExists{microtype.sty}{% use microtype if available
  \usepackage[]{microtype}
  \UseMicrotypeSet[protrusion]{basicmath} % disable protrusion for tt fonts
}{}
\makeatletter
\@ifundefined{KOMAClassName}{% if non-KOMA class
  \IfFileExists{parskip.sty}{%
    \usepackage{parskip}
  }{% else
    \setlength{\parindent}{0pt}
    \setlength{\parskip}{6pt plus 2pt minus 1pt}}
}{% if KOMA class
  \KOMAoptions{parskip=half}}
\makeatother
\usepackage{xcolor}
\IfFileExists{xurl.sty}{\usepackage{xurl}}{} % add URL line breaks if available
\IfFileExists{bookmark.sty}{\usepackage{bookmark}}{\usepackage{hyperref}}
\hypersetup{
  colorlinks=true,
  linkcolor={Maroon},
  filecolor={Maroon},
  citecolor={Blue},
  urlcolor={blue},
  pdfcreator={LaTeX via pandoc}}
\urlstyle{same} % disable monospaced font for URLs
\newif\ifbibliography
\usepackage{color}
\usepackage{fancyvrb}
\newcommand{\VerbBar}{|}
\newcommand{\VERB}{\Verb[commandchars=\\\{\}]}
\DefineVerbatimEnvironment{Highlighting}{Verbatim}{commandchars=\\\{\}}
% Add ',fontsize=\small' for more characters per line
\usepackage{framed}
\definecolor{shadecolor}{RGB}{248,248,248}
\newenvironment{Shaded}{\begin{snugshade}}{\end{snugshade}}
\newcommand{\AlertTok}[1]{\textcolor[rgb]{0.94,0.16,0.16}{#1}}
\newcommand{\AnnotationTok}[1]{\textcolor[rgb]{0.56,0.35,0.01}{\textbf{\textit{#1}}}}
\newcommand{\AttributeTok}[1]{\textcolor[rgb]{0.77,0.63,0.00}{#1}}
\newcommand{\BaseNTok}[1]{\textcolor[rgb]{0.00,0.00,0.81}{#1}}
\newcommand{\BuiltInTok}[1]{#1}
\newcommand{\CharTok}[1]{\textcolor[rgb]{0.31,0.60,0.02}{#1}}
\newcommand{\CommentTok}[1]{\textcolor[rgb]{0.56,0.35,0.01}{\textit{#1}}}
\newcommand{\CommentVarTok}[1]{\textcolor[rgb]{0.56,0.35,0.01}{\textbf{\textit{#1}}}}
\newcommand{\ConstantTok}[1]{\textcolor[rgb]{0.00,0.00,0.00}{#1}}
\newcommand{\ControlFlowTok}[1]{\textcolor[rgb]{0.13,0.29,0.53}{\textbf{#1}}}
\newcommand{\DataTypeTok}[1]{\textcolor[rgb]{0.13,0.29,0.53}{#1}}
\newcommand{\DecValTok}[1]{\textcolor[rgb]{0.00,0.00,0.81}{#1}}
\newcommand{\DocumentationTok}[1]{\textcolor[rgb]{0.56,0.35,0.01}{\textbf{\textit{#1}}}}
\newcommand{\ErrorTok}[1]{\textcolor[rgb]{0.64,0.00,0.00}{\textbf{#1}}}
\newcommand{\ExtensionTok}[1]{#1}
\newcommand{\FloatTok}[1]{\textcolor[rgb]{0.00,0.00,0.81}{#1}}
\newcommand{\FunctionTok}[1]{\textcolor[rgb]{0.00,0.00,0.00}{#1}}
\newcommand{\ImportTok}[1]{#1}
\newcommand{\InformationTok}[1]{\textcolor[rgb]{0.56,0.35,0.01}{\textbf{\textit{#1}}}}
\newcommand{\KeywordTok}[1]{\textcolor[rgb]{0.13,0.29,0.53}{\textbf{#1}}}
\newcommand{\NormalTok}[1]{#1}
\newcommand{\OperatorTok}[1]{\textcolor[rgb]{0.81,0.36,0.00}{\textbf{#1}}}
\newcommand{\OtherTok}[1]{\textcolor[rgb]{0.56,0.35,0.01}{#1}}
\newcommand{\PreprocessorTok}[1]{\textcolor[rgb]{0.56,0.35,0.01}{\textit{#1}}}
\newcommand{\RegionMarkerTok}[1]{#1}
\newcommand{\SpecialCharTok}[1]{\textcolor[rgb]{0.00,0.00,0.00}{#1}}
\newcommand{\SpecialStringTok}[1]{\textcolor[rgb]{0.31,0.60,0.02}{#1}}
\newcommand{\StringTok}[1]{\textcolor[rgb]{0.31,0.60,0.02}{#1}}
\newcommand{\VariableTok}[1]{\textcolor[rgb]{0.00,0.00,0.00}{#1}}
\newcommand{\VerbatimStringTok}[1]{\textcolor[rgb]{0.31,0.60,0.02}{#1}}
\newcommand{\WarningTok}[1]{\textcolor[rgb]{0.56,0.35,0.01}{\textbf{\textit{#1}}}}
\usepackage{longtable,booktabs,array}
\usepackage{calc} % for calculating minipage widths
\usepackage{caption}
% Make caption package work with longtable
\makeatletter
\def\fnum@table{\tablename~\thetable}
\makeatother
\usepackage{graphicx}
\makeatletter
\def\maxwidth{\ifdim\Gin@nat@width>\linewidth\linewidth\else\Gin@nat@width\fi}
\def\maxheight{\ifdim\Gin@nat@height>\textheight\textheight\else\Gin@nat@height\fi}
\makeatother
% Scale images if necessary, so that they will not overflow the page
% margins by default, and it is still possible to overwrite the defaults
% using explicit options in \includegraphics[width, height, ...]{}
\setkeys{Gin}{width=\maxwidth,height=\maxheight,keepaspectratio}
% Set default figure placement to htbp
\makeatletter
\def\fps@figure{htbp}
\makeatother
\setlength{\emergencystretch}{3em} % prevent overfull lines
\providecommand{\tightlist}{%
  \setlength{\itemsep}{0pt}\setlength{\parskip}{0pt}}
\setcounter{secnumdepth}{-\maxdimen} % remove section numbering
\setbeamertemplate{navigation symbols}{}
\setbeamertemplate{footline}[page number]
\ifLuaTeX
  \usepackage{selnolig}  % disable illegal ligatures
\fi

\begin{document}
\frame{\titlepage}

\begin{frame}[allowframebreaks]
  \tableofcontents[hideallsubsections]
\end{frame}
\hypertarget{graphiques}{%
\section{Graphiques}\label{graphiques}}

\begin{frame}[fragile]{Graphiques}
Cette section explique comment créer des types de graphique de base. La
commande la plus simple à utiliser pour représenter graphiquement un
ensemble de points est la commande \texttt{plot(x,y)}. La commande
\texttt{plot} a plusieurs arguments. Par défaut, cette commande trace
l'ensemble des points en points.

\begin{Shaded}
\begin{Highlighting}[]
\NormalTok{x}\OtherTok{=}\FunctionTok{c}\NormalTok{(}\SpecialCharTok{{-}}\DecValTok{2}\NormalTok{,}\DecValTok{1}\NormalTok{,}\DecValTok{5}\NormalTok{,}\SpecialCharTok{{-}}\DecValTok{4}\NormalTok{,}\DecValTok{0}\NormalTok{,}\DecValTok{3}\NormalTok{); }\FunctionTok{plot}\NormalTok{(x)}
\end{Highlighting}
\end{Shaded}

\includegraphics{Chap2_R_files/figure-beamer/unnamed-chunk-1-1.pdf}
\end{frame}

\begin{frame}[fragile]
Pour tracer une ligne, on doit ajouter l'argument \texttt{type="l"}.

\begin{Shaded}
\begin{Highlighting}[]
\FunctionTok{plot}\NormalTok{(x, }\AttributeTok{type=}\StringTok{"l"}\NormalTok{)}
\end{Highlighting}
\end{Shaded}

\includegraphics{Chap2_R_files/figure-beamer/unnamed-chunk-2-1.pdf}
\end{frame}

\begin{frame}[fragile]
Bien sûr, le logiciel \texttt{R} présente plusieurs arguments pour la
fonction plot, tels que le paramétrage des coleurs, largeur du trait de
la courbe, les étiquettes (labels) des axes, etc \ldots 

\begin{Shaded}
\begin{Highlighting}[]
\FunctionTok{plot}\NormalTok{(x,}\AttributeTok{type=}\StringTok{"l"}\NormalTok{, }\AttributeTok{col=}\StringTok{"red"}\NormalTok{,}\AttributeTok{lwd=}\DecValTok{2}\NormalTok{, }\AttributeTok{xlab=}\StringTok{"axes des abscisses"}\NormalTok{, }
\AttributeTok{ylab=}\StringTok{"axes des ordonnées"}\NormalTok{, }\AttributeTok{main=}\StringTok{"Mon premier graphique"}\NormalTok{, }
\AttributeTok{ylim=}\FunctionTok{c}\NormalTok{(}\SpecialCharTok{{-}}\DecValTok{6}\NormalTok{,}\DecValTok{6}\NormalTok{))}
\FunctionTok{points}\NormalTok{(x,}\AttributeTok{col=}\DecValTok{1}\SpecialCharTok{:}\DecValTok{6}\NormalTok{,}\AttributeTok{pch=}\DecValTok{1}\SpecialCharTok{:}\DecValTok{6}\NormalTok{,}\AttributeTok{lwd=}\DecValTok{3}\NormalTok{)}
\FunctionTok{legend}\NormalTok{(}\StringTok{"topleft"}\NormalTok{, }\StringTok{"Courbe"}\NormalTok{, }\AttributeTok{text.col=}\DecValTok{2}\NormalTok{, }\AttributeTok{lty=}\DecValTok{1}\NormalTok{)}
\end{Highlighting}
\end{Shaded}

\begin{center}\includegraphics{Chap2_R_files/figure-beamer/unnamed-chunk-3-1} \end{center}
\end{frame}

\begin{frame}[fragile]
La représentation des courbes des fonctions peut être faite de deux
manières; soit par la fonction \texttt{plot}, soit par la fonction
\texttt{curve}.

\begin{Shaded}
\begin{Highlighting}[]
\NormalTok{xx}\OtherTok{=}\FunctionTok{seq}\NormalTok{(}\SpecialCharTok{{-}}\DecValTok{2}\SpecialCharTok{*}\NormalTok{pi, }\DecValTok{2}\SpecialCharTok{*}\NormalTok{pi, }\AttributeTok{len=}\DecValTok{100}\NormalTok{); yy}\OtherTok{=} \FunctionTok{sin}\NormalTok{(xx)}
\FunctionTok{plot}\NormalTok{(xx,yy,}\AttributeTok{type=}\StringTok{"l"}\NormalTok{,}\AttributeTok{xlab=}\StringTok{"x"}\NormalTok{, }\AttributeTok{ylab=}\FunctionTok{expression}\NormalTok{(}\FunctionTok{sin}\NormalTok{(x)), }
     \AttributeTok{col=}\DecValTok{2}\NormalTok{, }\AttributeTok{lwd=}\DecValTok{3}\NormalTok{)}
\FunctionTok{curve}\NormalTok{(cos, }\SpecialCharTok{{-}}\DecValTok{2}\SpecialCharTok{*}\NormalTok{pi, }\DecValTok{2}\SpecialCharTok{*}\NormalTok{pi, }\AttributeTok{col=}\DecValTok{4}\NormalTok{, }\AttributeTok{lwd=}\DecValTok{3}\NormalTok{, }\AttributeTok{lty=}\DecValTok{2}\NormalTok{, }\AttributeTok{add=}\NormalTok{T)}
\FunctionTok{legend}\NormalTok{(}\StringTok{"bottomleft"}\NormalTok{,}\FunctionTok{c}\NormalTok{(}\StringTok{"sin"}\NormalTok{, }\StringTok{"cos"}\NormalTok{), }\AttributeTok{lty=}\FunctionTok{c}\NormalTok{(}\DecValTok{1}\NormalTok{,}\DecValTok{2}\NormalTok{), }\AttributeTok{col=}\FunctionTok{c}\NormalTok{(}\DecValTok{2}\NormalTok{,}\DecValTok{4}\NormalTok{), }
       \AttributeTok{lwd=}\DecValTok{3}\NormalTok{,}\AttributeTok{bty=}\StringTok{"n"}\NormalTok{)}
\end{Highlighting}
\end{Shaded}

\begin{center}\includegraphics{Chap2_R_files/figure-beamer/unnamed-chunk-4-1} \end{center}
\end{frame}

\begin{frame}{Les symboles graphiques}
\protect\hypertarget{les-symboles-graphiques}{}
La figure ci-dessous montre les différents types de points:

\begin{center}\includegraphics{Chap2_R_files/figure-beamer/unnamed-chunk-5-1} \end{center}
\end{frame}

\begin{frame}[fragile]{Les types des traits}
\protect\hypertarget{les-types-des-traits}{}
Le type de traits peut être spécifier en utilisant le paramètre
graphique \texttt{lty}. Les types de traits disponibles dans R sont :

\begin{center}\includegraphics{Chap2_R_files/figure-beamer/unnamed-chunk-6-1} \end{center}
\end{frame}

\begin{frame}[fragile]{Ajouter un texte}
\protect\hypertarget{ajouter-un-texte}{}
Pour ajouter du texte à un graphique avec le logiciel statistique R, les
fonctions \texttt{text()} et \texttt{mtext()} peuvent être utilisées.

\begin{Shaded}
\begin{Highlighting}[]
\FunctionTok{text}\NormalTok{(x,y,label)}
\end{Highlighting}
\end{Shaded}

x et y sont les coordonnées du texte à ajouter et label est le texte à
écrire sur le graphique.

\begin{Shaded}
\begin{Highlighting}[]
\NormalTok{x1}\OtherTok{=}\FunctionTok{cos}\NormalTok{(}\FunctionTok{seq}\NormalTok{(}\DecValTok{0}\NormalTok{,pi,}\AttributeTok{len=}\DecValTok{60}\NormalTok{)); }\FunctionTok{plot}\NormalTok{(x1,}\AttributeTok{type=}\StringTok{"n"}\NormalTok{, }\AttributeTok{xlab=}\StringTok{""}\NormalTok{, }\AttributeTok{ylab=}\StringTok{""}\NormalTok{)}
\FunctionTok{text}\NormalTok{(}\DecValTok{12}\NormalTok{, x1[}\DecValTok{12}\NormalTok{], }\StringTok{"Bonjour"}\NormalTok{, }\AttributeTok{col=}\DecValTok{2}\NormalTok{); }\FunctionTok{text}\NormalTok{(}\DecValTok{40}\NormalTok{, x1[}\DecValTok{40}\NormalTok{], }\StringTok{"Hello"}\NormalTok{, }\AttributeTok{col=}\DecValTok{4}\NormalTok{)}
\FunctionTok{text}\NormalTok{(}\DecValTok{50}\NormalTok{,}\FloatTok{0.5}\NormalTok{, }\FunctionTok{expression}\NormalTok{(}\FunctionTok{hat}\NormalTok{(beta)))}
\end{Highlighting}
\end{Shaded}

\begin{center}\includegraphics{Chap2_R_files/figure-beamer/unnamed-chunk-8-1} \end{center}
\end{frame}

\begin{frame}[fragile]{ggplot}
\protect\hypertarget{ggplot}{}
Une autre manière pour faire la représentation graphique est
l'utilisation de la fonction \texttt{ggplot} du package \texttt{ggplot2}
qui doit être installer par la commande
\texttt{install.package("ggplot2")}. Après l'installation, on fait appel
au package à l'aide \texttt{library("ggplot2")}.

\begin{Shaded}
\begin{Highlighting}[]
\FunctionTok{install.packages}\NormalTok{(}\StringTok{"ggplot2"}\NormalTok{)}
\FunctionTok{library}\NormalTok{(}\StringTok{"ggplot2"}\NormalTok{)}
\end{Highlighting}
\end{Shaded}

\pause

La fonction \texttt{ggplot} prend comme un premier argument une
\texttt{data.frame} qui contient les données à représenter. Un deuxième
argument \texttt{aes(x,y)} spécifie les valeurs des abscisses et les
ordonnées.
\end{frame}

\begin{frame}[fragile]
\begin{Shaded}
\begin{Highlighting}[]
\NormalTok{x}\OtherTok{=}\FunctionTok{seq}\NormalTok{(}\SpecialCharTok{{-}}\NormalTok{pi, pi, }\AttributeTok{len=}\DecValTok{100}\NormalTok{)}
\NormalTok{y}\OtherTok{=}\FunctionTok{sin}\NormalTok{(x); dd}\OtherTok{=}\FunctionTok{data.frame}\NormalTok{(x,y)}
\NormalTok{p}\OtherTok{=}\FunctionTok{ggplot}\NormalTok{(dd,}\FunctionTok{aes}\NormalTok{(x,y))}\SpecialCharTok{+}\FunctionTok{geom\_line}\NormalTok{()}
\NormalTok{p}
\end{Highlighting}
\end{Shaded}

\begin{center}\includegraphics[height=0.7\textheight]{Chap2_R_files/figure-beamer/unnamed-chunk-10-1} \end{center}
\end{frame}

\begin{frame}[fragile]
Les paramètres de la largeur et la couleur de la courbe doivent être
spécifiés dans \texttt{geom\_line()}. L'ajout d'un titre se fait par
l'ajout de \texttt{ggtitle()}.

\begin{Shaded}
\begin{Highlighting}[]
\NormalTok{p}\OtherTok{=}\NormalTok{p}\SpecialCharTok{+}\FunctionTok{geom\_line}\NormalTok{(}\AttributeTok{linewidth=}\FloatTok{1.2}\NormalTok{, }\AttributeTok{colour=}\StringTok{"blue"}\NormalTok{)}\SpecialCharTok{+} \FunctionTok{ggtitle}\NormalTok{(}\StringTok{"Titre"}\NormalTok{)}
\NormalTok{p}
\end{Highlighting}
\end{Shaded}

\begin{center}\includegraphics[height=0.7\textheight]{Chap2_R_files/figure-beamer/unnamed-chunk-11-1} \end{center}
\end{frame}

\begin{frame}[fragile]
Si on veut centrer le titre ou le mettre en couleur ou encore le mettre
en gras, on ajoutera
\texttt{theme(plot.title\ =\ element\_text(hjust\ =\ 0.5,\ size=20,\ color="darkred"))}.

\begin{Shaded}
\begin{Highlighting}[]
\NormalTok{p}\OtherTok{=}\NormalTok{p}\SpecialCharTok{+}\FunctionTok{theme}\NormalTok{(}\AttributeTok{plot.title =} \FunctionTok{element\_text}\NormalTok{(}\AttributeTok{hjust =} \FloatTok{0.5}\NormalTok{, }\AttributeTok{size=}\DecValTok{20}\NormalTok{,}
          \AttributeTok{color=}\StringTok{"darkred"}\NormalTok{))}\SpecialCharTok{+}\FunctionTok{labs}\NormalTok{(}\AttributeTok{x=}\StringTok{""}\NormalTok{,}\AttributeTok{y=}\StringTok{""}\NormalTok{)}
\NormalTok{p}
\end{Highlighting}
\end{Shaded}

\begin{center}\includegraphics[height=0.67\textheight]{Chap2_R_files/figure-beamer/unnamed-chunk-12-1} \end{center}
\end{frame}

\hypertarget{statistique-univariuxe9e}{%
\section{Statistique univariée}\label{statistique-univariuxe9e}}

\begin{frame}{Statistique univariée}
On entend par \(\color{red}{\textbf{statistique univariée}}\) l'étude
d'une seule variable, que celle-ci soit
\(\color{blue}{\textbf{qualitative}}\) ou
\(\color{blue}{\textbf{quantitative}}\). La statistique univariée fait
partie de la statistique descriptive.

\pause

Une \textbf{variable qualitative} (aussi appelée variable catégorique)
réfère à une caractéristique qui n'est pas quantifiable. Une variable
catégorique peut être nominale ou ordinale. \pause

\begin{itemize}
\item
  Une variable \(\color{violet}{\textbf{nominale}}\) décrit un nom, une
  étiquette ou une catégorie sans ordre naturel. Le sexe est un exemple.
\item
  Une variable \(\color{violet}{\textbf{ordinale}}\) est une variable
  dont les valeurs sont définies par une relation d'ordre entre les
  catégories possibles. La variable mention est une variable ordinale
  parce que la catégorie ``très bien'' est meilleure que la catégorie
  ``bien'' qui est meilleure de la catégorie ``passable''.\pause
\end{itemize}

Une \textbf{variable quantative} est une caractéristique quantifiable
dont les valeurs sont des nombres. Les variables numériques peuvent être
\(\color{blue}{\textbf{continues}}\) ou
\(\color{blue}{\textbf{discrètes}}\).\pause

\begin{itemize}
\item
  Variables continues: On dit qu'une variable est
  \(\color{violet}{\textbf{continue}}\) si elle prend un nombre infini
  de valeurs réelles possibles à l'intérieur d'un intervalle donné.
  Prenons la taille d'un élève par exemple.\pause
\item
  Variables discrètes: Contrairement à une variable continue, une
  variable \(\color{violet}{\textbf{discrète}}\) ne peut prendre qu'un
  nombre fini de valeurs réelles possibles à l'intérieur d'un intervalle
  donné. Le nombre d'enfants dans un ménage est un exemple.
\end{itemize}
\end{frame}

\hypertarget{variable-qualitative}{%
\section{Variable qualitative}\label{variable-qualitative}}

\begin{frame}{Variable qualitative}
Une variable qualitative peut être représentée, soit par un diagramme à
barres, soit par un diagramme en secteurs.\pause

\(\color{red}{\textbf{Exemple:}}\)

En 2005, les recettes du budget de l'État se présentaient de la façon
suivante (en milliards) :

\begin{longtable}[]{@{}lllllll@{}}
\toprule
\endhead
Source & TVA & IR & IS & TPP & AI & RNF \\
RF & 348 & 163 & 71 & 54 & 161 & 41 \\
\bottomrule
\end{longtable}

Le graphique à barres est parfois appelé graphique à bandes ou graphique
à bâtons. Il peut être horizontal ou vertical.

\vspace*{4cm}
\end{frame}

\begin{frame}[fragile]{Graphique à barres}
\protect\hypertarget{graphique-uxe0-barres}{}
\begin{block}{Diagramme à barres vertical}
\protect\hypertarget{diagramme-uxe0-barres-vertical}{}
\begin{Shaded}
\begin{Highlighting}[]
\NormalTok{RF}\OtherTok{=}\FunctionTok{c}\NormalTok{(}\DecValTok{348}\NormalTok{,}\DecValTok{163}\NormalTok{,}\DecValTok{71}\NormalTok{,}\DecValTok{54}\NormalTok{,}\DecValTok{161}\NormalTok{,}\DecValTok{41}\NormalTok{)}
\NormalTok{namess}\OtherTok{=}\FunctionTok{c}\NormalTok{(}\StringTok{"TVA"}\NormalTok{,}\StringTok{"IR"}\NormalTok{,}\StringTok{"IS"}\NormalTok{,}\StringTok{"TPP"}\NormalTok{,}\StringTok{"AI"}\NormalTok{,}\StringTok{"RNF"}\NormalTok{)}
\FunctionTok{barplot}\NormalTok{(RF, }\AttributeTok{names.arg=}\NormalTok{namess, }\AttributeTok{col=}\StringTok{"blue"}\NormalTok{, }\AttributeTok{ylim =} \FunctionTok{c}\NormalTok{(}\DecValTok{0}\NormalTok{,}\DecValTok{350}\NormalTok{))}
\end{Highlighting}
\end{Shaded}

\begin{center}\includegraphics[width=0.8\linewidth]{Chap2_R_files/figure-beamer/unnamed-chunk-14-1} \end{center}
\end{block}
\end{frame}

\begin{frame}[fragile]{Graphique à barres}
\protect\hypertarget{graphique-uxe0-barres-1}{}
\begin{block}{Diagramme à barres horizontal}
\protect\hypertarget{diagramme-uxe0-barres-horizontal}{}
\begin{Shaded}
\begin{Highlighting}[]
\FunctionTok{barplot}\NormalTok{(RF, }\AttributeTok{names.arg=}\NormalTok{namess, }\AttributeTok{col=}\StringTok{"blue"}\NormalTok{, }\AttributeTok{horiz =}\NormalTok{ T, }\AttributeTok{xlim =} \FunctionTok{c}\NormalTok{(}\DecValTok{0}\NormalTok{,}\DecValTok{350}\NormalTok{) )}
\end{Highlighting}
\end{Shaded}

\begin{center}\includegraphics[width=0.8\linewidth]{Chap2_R_files/figure-beamer/unnamed-chunk-15-1} \end{center}
\end{block}
\end{frame}

\begin{frame}[fragile]{Graphique à barres}
\protect\hypertarget{graphique-uxe0-barres-2}{}
\begin{block}{Diagramme à barres groupées}
\protect\hypertarget{diagramme-uxe0-barres-groupuxe9es}{}
\begin{Shaded}
\begin{Highlighting}[]
\NormalTok{cyl}\OtherTok{=}\FunctionTok{factor}\NormalTok{(mtcars}\SpecialCharTok{$}\NormalTok{cyl)}
\NormalTok{am}\OtherTok{=}\FunctionTok{factor}\NormalTok{(mtcars}\SpecialCharTok{$}\NormalTok{am, }\AttributeTok{labels =} \FunctionTok{c}\NormalTok{(}\StringTok{"Auto"}\NormalTok{, }\StringTok{"Manuel"}\NormalTok{))}
\NormalTok{tab}\OtherTok{=}\FunctionTok{table}\NormalTok{(am, cyl);}\FunctionTok{barplot}\NormalTok{(tab, }\AttributeTok{col=}\FunctionTok{c}\NormalTok{(}\StringTok{"lightblue"}\NormalTok{,}\StringTok{"blue2"}\NormalTok{))}
\FunctionTok{legend}\NormalTok{(}\StringTok{"top"}\NormalTok{, }\FunctionTok{c}\NormalTok{(}\StringTok{"Auto"}\NormalTok{,}\StringTok{"Manuel"}\NormalTok{), }\AttributeTok{fill=}\FunctionTok{c}\NormalTok{(}\StringTok{"lightblue"}\NormalTok{,}\StringTok{"blue2"}\NormalTok{), }\AttributeTok{box.lty =} \DecValTok{0}\NormalTok{)}
\end{Highlighting}
\end{Shaded}

\begin{center}\includegraphics[width=0.8\linewidth]{Chap2_R_files/figure-beamer/unnamed-chunk-16-1} \end{center}
\end{block}
\end{frame}

\begin{frame}[fragile]{Graphique à barres}
\protect\hypertarget{graphique-uxe0-barres-3}{}
\begin{block}{Diagramme à barres groupées}
\protect\hypertarget{diagramme-uxe0-barres-groupuxe9es-1}{}
\begin{Shaded}
\begin{Highlighting}[]
\FunctionTok{barplot}\NormalTok{(tab, }\AttributeTok{col=}\FunctionTok{c}\NormalTok{(}\StringTok{"lightblue"}\NormalTok{,}\StringTok{"blue2"}\NormalTok{), }\AttributeTok{beside=}\NormalTok{T)}
\FunctionTok{legend}\NormalTok{(}\StringTok{"top"}\NormalTok{, }\FunctionTok{c}\NormalTok{(}\StringTok{"Auto"}\NormalTok{,}\StringTok{"Manuel"}\NormalTok{), }\AttributeTok{fill=}\FunctionTok{c}\NormalTok{(}\StringTok{"lightblue"}\NormalTok{,}\StringTok{"blue2"}\NormalTok{), }\AttributeTok{box.lty =} \DecValTok{0}\NormalTok{)}
\end{Highlighting}
\end{Shaded}

\begin{center}\includegraphics[width=0.8\linewidth]{Chap2_R_files/figure-beamer/unnamed-chunk-17-1} \end{center}
\end{block}
\end{frame}

\begin{frame}[fragile]{Graphique circulaire}
\protect\hypertarget{graphique-circulaire}{}
Ce type de graphique est formé d'un cercle divisé en secteurs. Chaque
secteur représente une catégorie particulière.

\begin{block}{Diagramme simple}
\protect\hypertarget{diagramme-simple}{}
\begin{Shaded}
\begin{Highlighting}[]
\FunctionTok{pie}\NormalTok{(RF, namess, }\AttributeTok{col=}\FunctionTok{rainbow}\NormalTok{(}\FunctionTok{length}\NormalTok{(RF)))}
\end{Highlighting}
\end{Shaded}

\begin{center}\includegraphics[width=0.8\linewidth]{Chap2_R_files/figure-beamer/unnamed-chunk-18-1} \end{center}
\end{block}
\end{frame}

\begin{frame}[fragile]{Graphique circulaire}
\protect\hypertarget{graphique-circulaire-1}{}
Ce type de graphique est formé d'un cercle divisé en secteurs. Chaque
secteur représente une catégorie particulière.

\begin{block}{Diagramme 3D}
\protect\hypertarget{diagramme-3d}{}
\begin{Shaded}
\begin{Highlighting}[]
\NormalTok{plotrix}\SpecialCharTok{::}\FunctionTok{pie3D}\NormalTok{(RF, }\AttributeTok{labels=}\NormalTok{namess, }\AttributeTok{explode=}\FloatTok{0.1}\NormalTok{)}
\end{Highlighting}
\end{Shaded}

\begin{center}\includegraphics[width=0.8\linewidth]{Chap2_R_files/figure-beamer/unnamed-chunk-19-1} \end{center}
\end{block}
\end{frame}

\hypertarget{variable-quantitative}{%
\section{Variable quantitative}\label{variable-quantitative}}

\begin{frame}[fragile]{Variable quatitative discrète}
\protect\hypertarget{variable-quatitative-discruxe8te}{}
Une variable quantitative discrète peut être représentée graphiquement
par un graphique en bâton. Ce graphique s'obtient avec la fonction
\texttt{barplot()}.

\begin{Shaded}
\begin{Highlighting}[]
\FunctionTok{barplot}\NormalTok{(}\FunctionTok{table}\NormalTok{(mtcars}\SpecialCharTok{$}\NormalTok{cyl), }\AttributeTok{col=}\StringTok{"blue2"}\NormalTok{)}
\end{Highlighting}
\end{Shaded}

\begin{center}\includegraphics[width=0.8\linewidth]{Chap2_R_files/figure-beamer/unnamed-chunk-20-1} \end{center}
\end{frame}

\begin{frame}[fragile]{Variable quatitative continue}
\protect\hypertarget{variable-quatitative-continue}{}
Une variable quantitative continue peut être représentée par un
histogramme. Il est souvent employé pour montrer les caractéristiques
principales de la distribution des données de façon pratique. Ce type de
graphique s'obtient sous \texttt{R} \$ l'aide de la fonction
\texttt{boxplot}. \pause

\begin{Shaded}
\begin{Highlighting}[]
\FunctionTok{hist}\NormalTok{(mtcars}\SpecialCharTok{$}\NormalTok{mpg, }\AttributeTok{col=}\DecValTok{6}\NormalTok{,}\AttributeTok{probability =}\NormalTok{ T )}
\FunctionTok{lines}\NormalTok{(}\FunctionTok{density}\NormalTok{(mtcars}\SpecialCharTok{$}\NormalTok{mpg), }\AttributeTok{lwd=}\DecValTok{3}\NormalTok{)}
\end{Highlighting}
\end{Shaded}

\begin{center}\includegraphics[width=0.8\linewidth]{Chap2_R_files/figure-beamer/unnamed-chunk-21-1} \end{center}
\end{frame}

\begin{frame}[fragile]{Variable quatitative continue}
\protect\hypertarget{variable-quatitative-continue-1}{}
On peut aussi résumer une variable statistique quantitative par un
diagramme appelé boite à moustache (boxplot).

\begin{Shaded}
\begin{Highlighting}[]
\FunctionTok{boxplot}\NormalTok{(mtcars}\SpecialCharTok{$}\NormalTok{mpg, }\AttributeTok{col=}\DecValTok{5}\NormalTok{)}
\end{Highlighting}
\end{Shaded}

\begin{center}\includegraphics[width=0.8\linewidth]{Chap2_R_files/figure-beamer/unnamed-chunk-22-1} \end{center}
\end{frame}

\begin{frame}[fragile]{Variable quatitative continue}
\protect\hypertarget{variable-quatitative-continue-2}{}
\begin{Shaded}
\begin{Highlighting}[]
\FunctionTok{boxplot}\NormalTok{(mpg }\SpecialCharTok{\textasciitilde{}}\NormalTok{ cyl , }\AttributeTok{data=}\NormalTok{mtcars, }\AttributeTok{col=}\DecValTok{5}\NormalTok{, }\AttributeTok{notch=}\NormalTok{T, }\AttributeTok{horizontal =}\NormalTok{ T)}
\end{Highlighting}
\end{Shaded}

\begin{center}\includegraphics[width=0.8\linewidth]{Chap2_R_files/figure-beamer/unnamed-chunk-23-1} \end{center}
\end{frame}

\hypertarget{analyse-statistique}{%
\section{Analyse statistique}\label{analyse-statistique}}

\begin{frame}[fragile]{Variable qualitative}
\protect\hypertarget{variable-qualitative-1}{}
Pour une variable qualitative, on peut déterminer un tableau des
fréquences (absolues ou relatives).

\begin{Shaded}
\begin{Highlighting}[]
\NormalTok{(}\AttributeTok{t1=}\FunctionTok{table}\NormalTok{(mtcars}\SpecialCharTok{$}\NormalTok{cyl)) }\CommentTok{\# absolues}
\end{Highlighting}
\end{Shaded}

\begin{verbatim}
 4  6  8 
11  7 14 
\end{verbatim}

\begin{Shaded}
\begin{Highlighting}[]
\NormalTok{t1}\SpecialCharTok{/}\FunctionTok{sum}\NormalTok{(t1)             }\CommentTok{\# relatives}
\end{Highlighting}
\end{Shaded}

\begin{verbatim}
      4       6       8 
0.34375 0.21875 0.43750 
\end{verbatim}

\begin{Shaded}
\begin{Highlighting}[]
\NormalTok{(}\AttributeTok{t2=}\FunctionTok{table}\NormalTok{(mtcars}\SpecialCharTok{$}\NormalTok{cyl, mtcars}\SpecialCharTok{$}\NormalTok{gear )); t2}\SpecialCharTok{/}\FunctionTok{sum}\NormalTok{(t2)}
\end{Highlighting}
\end{Shaded}

\begin{verbatim}
   
     3  4  5
  4  1  8  2
  6  2  4  1
  8 12  0  2
\end{verbatim}

\begin{verbatim}
   
          3       4       5
  4 0.03125 0.25000 0.06250
  6 0.06250 0.12500 0.03125
  8 0.37500 0.00000 0.06250
\end{verbatim}
\end{frame}

\begin{frame}[fragile]{Variable quantitative}
\protect\hypertarget{variable-quantitative-1}{}
Une première analyse statistique pour une variable quantitative est
l'analyse descriptive.

\begin{Shaded}
\begin{Highlighting}[]
\FunctionTok{summary}\NormalTok{(mtcars}\SpecialCharTok{$}\NormalTok{mpg)}
\end{Highlighting}
\end{Shaded}

\begin{verbatim}
   Min. 1st Qu.  Median    Mean 3rd Qu.    Max. 
  10.40   15.43   19.20   20.09   22.80   33.90 
\end{verbatim}

\begin{Shaded}
\begin{Highlighting}[]
\NormalTok{pastecs}\SpecialCharTok{::}\FunctionTok{stat.desc}\NormalTok{(mtcars}\SpecialCharTok{$}\NormalTok{mpg, }\AttributeTok{norm=}\NormalTok{T, }\AttributeTok{basic=}\NormalTok{F )}
\end{Highlighting}
\end{Shaded}

\begin{verbatim}
      median         mean      SE.mean CI.mean.0.95          var      std.dev 
  19.2000000   20.0906250    1.0654240    2.1729465   36.3241028    6.0269481 
    coef.var     skewness     skew.2SE     kurtosis     kurt.2SE   normtest.W 
   0.2999881    0.6106550    0.7366922   -0.3727660   -0.2302812    0.9475647 
  normtest.p 
   0.1228814 
\end{verbatim}

\vspace*{2cm}
\end{frame}

\end{document}
